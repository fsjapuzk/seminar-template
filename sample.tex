\documentclass{japuzk}

\title{Titel der Arbeit}
\subtitle{Titel des Moduls}
\date{[Datum]}
\author{Vorname Nachname}
\japdocmoduletitle{Titel des Moduls}
\japdoctype{Hausarbeit / Essay}
\japdoclvtitle{Titel der Lehrveranstaltung}
\japdocsemester{Winter / Sommersemester 20XX}
\japdocsupervisor{(Prof. Dr. / Dr.) Name des Dozierenden}
\japdocmatriculation{Matrikelnummer}
\japdocstudysemester{XX}
\japdocfieldofstudy{Studiengang}
\japdocemail{E-Mail-Adresse}
\japdocaddress{Straße Hausnummer, PLZ Ort}

\usepackage{csquotes}
\usepackage{hyperref}
\usepackage{enumitem}
\setlist[description]{style=nextline}

\newcommand{\bibresourcename}{sample.bib}
\addbibresource{\bibresourcename}

\japterm{kanji}{漢字}
\japtermId{nengou}{nengô}{年号}
\japtermtranslateId{shoujo-manga}{shôjo manga}{少女マンガ}{Mädchencomic}
\japtermnameId[pre]{natsume}{Natsume}{Sôseki}{夏目漱石}{1867}{1916}
\japtermeraId{meiji}{Meiji}{1868}{1912}

\begin{document}

\maketitle
\tableofcontents

\section{Vorbemerkung}
Diese Musterdatei soll Ihnen die formale Ausgestaltung des Hauptteils einer Hausarbeit anhand von Beispielen vor Augen führen. Die vollständigen Vorgaben finden Sie im Überblick auf der \href{\japfacultyhomepage}{Homepage der Japanologie}.

\section{Formales}
Der linke Seitenrand ist 2 cm breit, der rechte 3 cm. Oben beträgt der Abstand 2,5 cm und unten 2 cm. Die verwendete Schriftart heißt \enquote{Times New Roman}, es handelt sich dabei um eine Serifenschrift. Die Schriftgröße beträgt 12 pt, der Zeilenabstand ist auf 16 pt festgesetzt.\footnote{In Fußnoten wird die gleiche Schriftart wie im Haupttext verwendet, allerdings in der Schriftgröße 10 pt. Der Zeilenabstand beträgt hier genau 13 pt. Fußnoten enthalten z.B. Quellen, Verweise, Zusatzinformationen etc. Sie beginnen immer mit einem Großbuchstaben und enden mit einem Punkt.}

Die wissenschaftliche Hausarbeit ist in \textbf{einfacher Ausfertigung} ausgedruckt und geheftet im Sekretariat einzureichen. Zusätzlich dazu muss der verantwortlichen Lehrperson ein Exemplar im \textbf{Word- oder OpenOffice-Format} per E-Mail zugehen. Die Abgabetermine werden rechtzeitig bekanntgeben und/oder durch Aushänge mitgeteilt.

\section{Der Hauptteil}
Eine wissenschaftliche Arbeit stellt zusammenhängende Gedankengänge und Argumentationen dar. Absätze bestehen daher aus mindestens zwei Sätzen und mehr!

Beim erstmaligen Auftreten eines japanischsprachigen Begriffes sind die Schriftzeichen mit anzuführen, ferner noch bei Personen die Lebensdaten, bei Werken, Filmen etc. die deutsche Übersetzung und das Entstehungsjahr, bei Jahresdevisen (\japref{nengou}) die Jahresangaben etc. Transliteration und Schriftzeichen sind dabei nicht zu trennen.\newline
\textbf{Beispiele}: \japref{shoujo-manga}; \japref{natsume}, \japref{meiji} etc.\newline
Verwenden Sie für die korrekte Transliteration (=Umschrift) unbedingt die Richtlinien auf der Homepage der Japanologie.

\subsection{Zitieren und Belegen}
Wesentliches Beurteilungskriterium für wissenschaftliches Arbeiten ist das Offenlegen der verwendeten Quellen. Das konstruktive Arbeiten mit Quellen (direkte Zitate, Verweise etc.) zeigt Ihre erworbene/erlesene Fachkompetenz auf dem Gebiet. Eine bloße Auflistung von Titeln im Literaturverzeichnis stellt kein wissenschaftliches Arbeiten dar. Der Verweis auf andere Quellen ist kein Mangel an \enquote{Kreativität}, sondern belegt Ihre Belesenheit. Machen Sie von den unterschiedlichen Möglichkeiten der Informationswiedergabe Gebrauch. Direkte und indirekte Zitate erfolgen immer aus der Originalquelle. Das direkte oder indirekte Zitieren eines Autors aus einer anderen wissenschaftlichen Arbeit ist ein \enquote{Drittzitat} und als solches nicht legitim.

\subsubsection{Direktes Zitat}
Direkte oder wörtliche Zitate werden im Text durch Anführungszeichen gekennzeichnet und müssen durch eine eindeutige Quellenangabe mit Seitenzahl in der Fußnote ausgewiesen werden.\newline
\textbf{Beispiel}: \citeauthor{muster:quellen} sagt hierzu: \enquote{Direkte Zitate ohne Fußnote und Quellenangabe sind wertlos.}\footcite[12]{muster:quellen}

\subsubsection{Indirektes Zitat}
Indirekte Zitate sind Wiedergaben einer Aussage Dritter (z.B. Der Autor meint, dass~\ldots~sei; oder bestimmte Verben wie \enquote{gelten}, \enquote{sollen}, \enquote{scheinen}), die als solches ebenfalls durch eine Quellenangabe mit Seitenzahl in der Fußnote ausgewiesen werden müssen.\newline
\textbf{Beispiel}: Eine hiervon abweichende Praxis sieht \citeauthor{doe:references} als Missachtung wissenschaftlicher Prinzipien an.\footcite[Vgl.][13\psq]{doe:references}

\subsubsection{Drittzitat}
Sogenannte Drittzitate, auch als Zweitzitate, Sekundärzitate oder Zitate aus zweiter Hand bezeichnet, sind Inhalte oder wörtliche Zitate aus Publikationen, die ungeprüft, ohne den zugrundeliegenden Text gelesen zu haben, in die eigene Arbeit übernommen werden. \underline{Dieses Vorgehen ist wissenschaftlich nicht zulässig!}\newline
Wenn beispielsweise in einem Text, der Ihnen vorliegt, ein direktes Zitat von \japref{natsume} angeführt wird, dürfen Sie dieses \japref{natsume}-Zitat nicht übernehmen, sondern Sie müssen im Originaltext von \japref{natsume} nachschauen und direkt daraus zitieren.

\subsection{Form der Quellenangaben}
Beim erstmaligen Verweis auf eine Quelle in der Fußnote sind die \textbf{vollständigen Literaturangaben} aufzuführen. Beachten Sie hierfür unsere Vorgaben für die Notation! Bei allen weiteren Nennungen wird ein Kurzbeleg (Autor, (Kurz-)Titel und Seitenzahl) verwendet.\footcite[Vgl.][10]{muster:quellen}

\subsubsection{Japanische Quellen}
Bei Zitaten aus dem Japanischen muss der Originaltext nicht noch einmal wiedergegeben werden, die genaue Quellenangabe reicht hier aus. Bei japanischen Titeln werden Autor, Titel, Reihe und Zeitschriftentitel transliteriert und mit den entsprechenden \japref{kanji} versehen. Die \japref{kanji} der Buchtitel werden \textbf{nicht} kursiv gesetzt! Eine Übersetzung des Titels ist in der Regel nicht nötig, es sei denn, im Fließtext wird explizit darauf Bezug genommen. Verlagsort und Verlag können ohne Angabe der \japref{kanji} lediglich transliteriert werden.\footcite[Vgl.][112]{suzuki:ronbun}

\printbibliography[heading=bibintoc]

\appendix
\newrefsection

\section{Zitationsdaten}

Hier noch einige Beispiele für Zitationsdaten entsprechend der Angaben im Leitfaden. Die entsprechenden Bib(La)TeX-Felder können der beigefügten \texttt{\bibresourcename} Datei entnommen werden.

Internetquellen werden zurzeit noch nicht (korrekt) unterstützt. Entsprechende Felder für die URL sind im BibTeX-Standard zwar vorhanden und können in die Bibliographie-Datei eingebunden werden;\ es wird jedoch kein Gewähr dafür übernommen, dass diese auch korrekt dargestellt werden.

\subsection{Monographien}

\textit{Hinweis}: Diese Formatbeispiele enden bewusst \emph{nicht} mit einem Punkt. BibTeX unterscheidet nämlich zwischen \texttt{bibliography}- und \texttt{cite}-Stil. Ersterer wird im Literaturverzeichnis verwendet und enthält standardmäßig einen Punkt. Letzterer wird im Fließtext verwendet und enthält keinen Punkt, um die Einbindung eines Zitats in einen Satz zu ermöglichen. Punkte am Ende müssen ggf.\ manuell nachgesetzt werden.

\begin{description}
	\item[Monographie mit 1 Autor]\fullcite{monogr:autor}
	\item[Monographie mit 1 Herausgeber]\fullcite{monogr:hrsg}
	\item[Monographie mit 1 Autor innerhalb einer Reihe]\fullcite{monogr:autor-reihe}
	\item[Monographie mit 1 Herausgeber innerhalb einer Reihe]\fullcite{monogr:hrsg-reihe}
\end{description}

\underline{Anmerkung}: Bei bis zu drei Autoren, Herausgebern oder Verlagsorten werden alle angeführt; bei mehr als drei wird nur der erste genannt und dahinter mit \enquote{u.a.} abgekürzt.

\begin{itemize}
	\item \fullcite{monogr:autoren-drei}
	\item \fullcite{monogr:autoren-viele}
\end{itemize}

\subsection{Sammelbände und Zeitschriften}

\begin{description}
	\item[Artikel/Beitrag in einem Sammelband]\fullcite{artikel:sammelband}
	\item[Artikel/Beitrag in einer Zeitschrift]\fullcite{artikel:zeitschrift}
\end{description}

\underline{Anmerkung}: Bei bis zu drei Autoren, Herausgebern oder Verlagsorten werden alle angeführt, bei mehr als drei wird nur der erste genannt und dahinter mit \enquote{u.a.} abgekürzt.

\begin{itemize}
	\item \fullcite{artikel:autoren-drei}
	\item \fullcite{artikel:autoren-viele}
\end{itemize}

\subsection{Japanischsprachige Literatur}

Bib(La)Tex unterstützt (noch) keine Wiedergabe von Autorennamen in verschiedenen Schriftsystemen. Daher muss für die Unterstützung in der aktuellen Formatvorlage eine Behelfslösung genutzt werden: Der bibliographischen Datenbank wird Glauben gemacht, der Mensch heiße \enquote{Nachname} mit Nachnamen und \enquote{Vorname Kanji Kanji} mit Vornamen. Konkret resultiert dies in folgender, beispielhafter BibTeX-Angabe:

\begin{verbatim}
author = {Suzuki, {Kazuko 鈴木和子} and Yamada, {Ken'ichi 山田健一}}
\end{verbatim}

Insbesondere werden also die Kanji für den Vor- \textbf{und} Nachnamen mit geschweiften Klammern an den romanisierten Vornamen gebunden. Eine Kursivsetzung erfolgt dabei nur \enquote{aus Glück} nicht, da \texttt{luatex-ja} keine kursiven Fonts für Kanji kennt.

Mit Publikationstiteln wird analog verfahren. Alternativ kann hier auch das Feld \texttt{titleaddon} belegt werden.

\begin{verbatim}
title = {Ronbun wo kakimashô! 論文を書きましょう!}
\end{verbatim}

\begin{description}
	\item[Beispiel japanische Monographie] \fullcite{jap:monogr}
	\item[Beispiel japanischer Zeitschriftenartikel] \fullcite{jap:artikel}
\end{description}

\printbibliography[heading=bibintoc]

\end{document}